\setbeamertemplate{block begin}{
  \vskip.75ex
  \begin{beamercolorbox}[rounded=true,shadow=true,leftskip=1cm,colsep*=.75ex]{block title}%
    \usebeamerfont*{block title}\insertblocktitle
  \end{beamercolorbox}%
  {\ifbeamercolorempty[bg]{block body}{}{\nointerlineskip\vskip-0.5pt}}%
  \usebeamerfont{block body}%
  \begin{beamercolorbox}[ht=19.33cm, rounded=true,shadow=true,colsep*=.75ex,sep=.75ex, vmode]{block body}%
    \ifbeamercolorempty[bg]{block body}{\vskip-.25ex}{\vskip-.75ex}\vbox{}%
    \begin{adjustwidth}{0.8cm}{0.4cm}
  }
  \setbeamertemplate{block end}{
  \end{adjustwidth}
  \end{beamercolorbox}
}
\begin{block}{\large \smash{5. The linear format: LPPE}\vphantom{Introduction}}
We define \alert{LPPEs} (linear probabilistic process equations) as follows:
\vskip-25pt
\begin{align*}
X(\vec{g\vphantom{G}}:\vec{G}) = & \sum_{\vec{d_1} : \vec{D_1}} c_1 \Rightarrow a_1(b_1) \!\psum_{\vec{\vphantom{E}e_1} : \vec{E_1}} f_1 \colon X(n_1)\\
& \dots\\[10pt]
 + & \sum_{\vec{d_k} : \vec{D_k}} c_k \Rightarrow a_k(b_k) \!\psum_{\vec{\vphantom{E}e_k} : \vec{E_k}} f_k \colon X(n_k)
\end{align*}

\vskip40pt

Advantages of LPPEs:
\begin{adjustwidth}{0.5cm}{0.5cm}
\begin{itemize}
\item \ The \alert{state space} can be generated very easily
\item \ \alert{Parallel composition} can be applied in a straight-forward manner
\item \ \alert{Symbolic optimisations} are enabled at the language level
\end{itemize}
\end{adjustwidth}
\end{block}	