\chapter{Related work}
There are countless simulators available for various embedded architectures.
However, there do not appear to be many simulator generators around. The only
other project that we have been able to find is CGEN, which stands for
Cpu tools GENerator. It can generate simulators, assemblers and disassemblers. 
The generator is implemented in Scheme, which is also the
language used to write the specification of the Instruction Set Architecture,
"inspired by GCC's RTL and by the Scheme programming language, theoretically
taking the best of both" \cite{CGEN}.\\
Currently the opcodes table and an ISA level simulator need to be specified, 
but the goal is to have "an application independent description of the CPU".\\
The documentation claims good speed by implemented a threaded interpreter with
the use of computed gotos, where "about three host instructions per target
instruction" are needed to jump to the next instruction handler. A typical
instruction handler takes about ten host instructions.\\
\\
CGEN is used in some other projects, such as SID and GDB. SID is a "framework
for building computer system simulations" \cite{SID}. GDB is the GNU DeBugger,
which can be used to debug a variety of languages on a lot of target
platforms \cite{GDB}. Currently several of
simulators provided by GDB use CGEN. "These simulators use
a combination of the CGEN-generated files, common files in the sim/common
directory of GDB releases, and custom target-specific code". This
functionality can be used by issuing 
\texttt{target \textless sim\textgreater} from the GDB command
line (if built with \texttt{--enable-sim}), after which programs can be loaded
and run in the usual manner.
