\chapter{User guide}
\section{Basics}
\subsection{Style}
The \ac{ESSIG} specification style is influenced by Java. The biggest
difference can be found in typing which we shall see later on. However, both
separate lines by a semi-colon and it can also can be seen in blocks which have
the following pattern:

\begin{lstlisting}
IDENTIFIER {
	line1;
	line2;
	...
}
\end{lstlisting}


\subsection[IDENTIFIER]{Identifiers}
Identifiers are names and must start with a letter and may be followed with
more letters and digits:

\begin{lstlisting}
fragment DIGIT	:	'0'..'9';
fragment LETTER	:	('a'..'z' | 'A'..'Z');
IDENTIFIER		:	LETTER (LETTER | DIGIT)*;
\end{lstlisting}

\subsection{Numbers}
Numbers can be specified either as numbers with an optional sign to indicate it
is a negative number or hexadecimal numbers.

\begin{lstlisting}
fragment DIGIT	:	'0'..'9';
fragment NUM	:	'-'? DIGIT+;
fragment HEX_NUM:	'0' 'x' (DIGIT | ('a'..'f'|'A'..'F'))+;
NUMBER			:	(NUM | HEX_NUM);
\end{lstlisting}

Example:
\begin{lstlisting}
10		// Number
-7		// Negative number
0x15f	// Hexadecimal number
\end{lstlisting}

\subsection{Comments}
Comments are C-style:

\begin{lstlisting}
COMMENT		:	'//' ~('\n'|'\r')* '\r'? '\n'
			|	'/*' .* '*/'
			;
\end{lstlisting}

Example:
\begin{lstlisting}
// Single line comment
/*
multi
line
comment
*/
\end{lstlisting}

\subsection{Opcodes}
\begin{lstlisting}
OPCODE		:	'"' ('0' | '1' | LETTER | ' ')+ '"';
\end{lstlisting}

Example:
\begin{lstlisting}
"0000 0000 0000 0000" // Fixed opcode
"00dd dd11 01rr ddrr" // 4-bit variable d and 4-bit variable r
\end{lstlisting}

\section{Specifying a microcontroller}
In order to specify a microcontroller, we have the basic block pattern with an
identifier as a name and must contain four blocks:
\begin{lstlisting}
microcontroller:	IDENTIFIER '{'
						parameters
						registers
						maps
						instructions
					'}'
			;
\end{lstlisting}

\subsection{Parameters}
Parameters follow the basic block pattern with a parameter specification on
each line.

\begin{lstlisting}
parameters	:	'parameters' '{' (parameter ';')+ '}';

parameter	:	'gprs' NUMBER ('+' NUMBER)?
			|	'opcode-size' NUMBER
			|	'clock' NUMBER
			|	'endianness' ('big' | 'little')
			;
\end{lstlisting}

Example:
\begin{lstlisting}
parameters {
	gprs 32+5; // 32 general purpose registers at offset 5
	opcode-size 16; // 16 bits opcodes
	clock 1; // Standard 1 clock cycle
	endianness little; // Little endian
}
\end{lstlisting}

It should be noted that each parameter has specific behaviour.

\subsection{Registers}
Registers are specified as offsets. Remember that the gprs-parameter also
specifies registers and you will be warned when they overlap.
\begin{lstlisting}
registers	:	'registers' '{' (register ';')+ '}'

register	:	IDENTIFIER '=' NUMBER;
\end{lstlisting}

Example:
\begin{lstlisting}
registers {
	pc = 10;
	sreg = 11;
}
\end{lstlisting}

\subsection{Maps}
Maps are memory mappings. All values are absolute and can overlap thus mapping
IO into memory space.
\begin{lstlisting}
MAP_TYPE	:	'chunk' | 'io' | 'ram' | 'rom' | 'register' | 'print';

maps		:	'maps' '{' (map ';')+ '}';

map			:	MAP_TYPE '(' NUMBER ';' NUMBER ')';
\end{lstlisting}

Example:
\begin{lstlisting}
maps {
	chunk		(0, 0xFFFFFFF);
	register	(0, 0x20);
	io			(0x20, 0x60);
	ram			(0, 0x461);
	rom			(0x464,0xFFFFFF);
	print		(0x3b, 0x3c);
}
\end{lstlisting}

\subsection{Instructions}
Instructions are the most complex block. First of all we'll start with the
basics and handle expressions later.

\begin{lstlisting}
OPCODE		:	'"' (DIGIT | LETTER | ' ')+ '"';

params		:	OPCODE (',' OPCODE)* (',' 'clock' '=' NUMBER)?

arguments	:		argument (',' argument)*;

argument	:		'~'? identifier;

instructions:	'instructions' '{' instruction+ '}';

instruction	:	IDENTIFIER params arguments? '{' expr+ '}';
\end{lstlisting}

Example:
\begin{lstlisting}
instructions {
	noop "0000 0000 0000 0000" {
		// Expressions here
	}

	adc "0001 11rd dddd rrrr" Rd, Rr {
		// Expressions here
	}

	rjmp "1100 kkkk kkkk kkkk" ~k {
		// Expressions here
	}
}
\end{lstlisting}

\subsection{Expressions}
\begin{lstlisting}
variable	:	'$'? IDENTIFIER
			|	MAP_TYPE '{' operatorExpr '}'
			|	( IDENTIFIER | MAP_TYPE ) '[' operatorExpr ':' operatorExpr ']'
			;

word 		:	variable ('.' (NUMBER | constant))?
			|	NUMBER
			|	'!' word
			;

operatorExpr:	word (('&' | '|' | '^' | '+' | '-' | '*' | '<<' | '>>') operatorExpr)?;

assignExpr	:	variable '=' operatorExpr;

comparison	:	'==' | '<' | '=<' | '>' | '>=';

condition	:	word comparison word
			|	'(' operatorExpr ')' comparison word
			;

ifExpr		:	'if' condition '{' expr+ '}' ('else' '{' expr+ '}')?;

expr		:	assignExpr ';'
			|	ifExpr
			|	'HALT' ';'
			;
\end{lstlisting}

Example:
\begin{lstlisting}
// ifExpr
if $k == -1 {
	HALT;
} else {
	// assignExpr
	PC = PC + $k + 1;
}

// assignExpr: write to ram location ram[$SP+1..$SP] the value PC + 1
ram(ram[$SP+1..$SP]) = PC + 1;

// Decrease ram[$SP+1..$SP] by 2
ram[$SP+1..$SP] = ram[$SP+1..$SP] - 2;

// FIXME More examples
\end{lstlisting}
