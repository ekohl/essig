\chapter{User guide}

\section{Installation}
The code can be retrieved from \texttt{https://github.com/fmt-essig/essig}, or
checked out directly: \texttt{git clone git://github.com/fmt-essig/essig.git}.
Then see README.rst in the project's root directory for instructions.

See grammar/examples for example specifications.

\section{Basics}
\subsection{Style}
The \ac{ESSIG} specification style is influenced by Java. The biggest
difference can be found in typing which we shall see later on. However, both
separate lines by a semi-colon and it can also can be seen in blocks which have
the following pattern:

\lstset{caption=Block specification pattern}
\begin{lstlisting}
IDENTIFIER {
	line1;
	line2;
	...
}
\end{lstlisting}


\subsection[IDENTIFIER]{Identifiers}
Identifiers are names and must start with a letter and may be followed with
more letters and digits:

\lstset{caption=Identifier specifation}
\begin{lstlisting}
fragment DIGIT	:	'0'..'9';
fragment LETTER	:	('a'..'z' | 'A'..'Z');
IDENTIFIER		:	LETTER (LETTER | DIGIT)*;
\end{lstlisting}

\subsection{Numbers}
Numbers can be specified either as numbers with an optional sign to indicate it
is a negative number or hexadecimal numbers.

\lstset{caption=Number specifation}
\begin{lstlisting}
fragment DIGIT	:	'0'..'9';
fragment NUM	:	'-'? DIGIT+;
fragment HEX_NUM:	'0' 'x' (DIGIT | ('a'..'f'|'A'..'F'))+;
NUMBER			:	(NUM | HEX_NUM);
\end{lstlisting}

\lstset{caption=Example numbers}
\begin{lstlisting}
10		// Number
-7		// Negative number
0x15f	// Hexadecimal number
\end{lstlisting}

\subsection{Comments}
Comments are C-style:

\lstset{caption=Comment specifation}
\begin{lstlisting}
COMMENT		:	'//' ~('\n'|'\r')* '\r'? '\n'
			|	'/*' .* '*/'
			;
\end{lstlisting}

\lstset{caption=Example comments}
\begin{lstlisting}
// Single line comment
/*
multi
line
comment
*/
\end{lstlisting}

\section{Specifying a microcontroller}
In order to specify a microcontroller, we have the basic block pattern with an
identifier as a name and must contain four blocks:
\lstset{caption=Microcontroller specification}
\begin{lstlisting}
microcontroller:	IDENTIFIER '{'
						parameters
						registers
						maps
						instructions
					'}'
			;
\end{lstlisting}

\subsection{Parameters}
Parameters follow the basic block pattern with a parameter specification on
each line.

\lstset{caption=Parameter specification}
\begin{lstlisting}
parameters	:	'parameters' '{' (parameter ';')+ '}';

parameter	:	'gprs' NUMBER ('+' NUMBER)?
			|	'opcode-size' NUMBER
			|	'clock' NUMBER
			|	'endianness' ('big' | 'little')
			;
\end{lstlisting}

\lstset{caption=Example parameters}
\begin{lstlisting}
parameters {
	gprs 32+5; // 32 general purpose registers at offset 5
	opcode-size 16; // 16 bits opcodes
	clock 1; // Standard 1 clock cycle
	endianness little; // Little endian
}
\end{lstlisting}

It should be noted that each parameter has specific behaviour.

\subsection{Registers}
Registers are specified as offsets. Remember that the gprs-parameter also
specifies registers and you will be warned when they overlap.
\lstset{caption=Register specification}
\begin{lstlisting}
registers	:	'registers' '{' (register ';')+ '}'

register	:	IDENTIFIER '=' NUMBER;
\end{lstlisting}

\lstset{caption=Example register specification}
\begin{lstlisting}
registers {
	pc = 10;
	sreg = 11;
}
\end{lstlisting}

\subsection{Maps}
Maps are memory mappings. All values are absolute and can overlap thus mapping
IO into memory space.
\lstset{caption=Memory mapping specification}
\begin{lstlisting}
MAP_TYPE	:	'chunk' | 'io' | 'ram' | 'rom' | 'register' | 'print';

maps		:	'maps' '{' (map ';')+ '}';

map			:	MAP_TYPE '(' NUMBER ';' NUMBER ')';
\end{lstlisting}

\lstset{caption=Example memory mapping}
\begin{lstlisting}
maps {
	chunk		(0, 0xFFFFFFF);
	register	(0, 0x20);
	io			(0x20, 0x60);
	ram			(0, 0x461);
	rom			(0x464,0xFFFFFF);
	print		(0x3b, 0x3c);
}
\end{lstlisting}

\subsection{Instructions}
Instructions are the most complex block. First of all we'll start with the
basics and handle expressions later.
\lstset{caption=Instruction specification}
\begin{lstlisting}
OPCODE		:	'"' ('0' | '1' | LETTER | ' ')+ '"';

params		:	OPCODE (',' OPCODE)* (',' 'clock' '=' NUMBER)?

arguments	:	argument (',' argument)*;

argument	:	'~'? identifier;

instructions:	'instructions' '{' instruction+ '}';

instruction	:	IDENTIFIER params arguments? '{' expr+ '}';
\end{lstlisting}

\lstset{caption=Example instructions}
\begin{lstlisting}
instructions {
	noop "0000 0000 0000 0000" {
		// Expressions here
	}

	adc "0001 11rd dddd rrrr" Rd, Rr {
		// Expressions here
	}

	rjmp "1100 kkkk kkkk kkkk", clock=2 ~k {
		// Expressions here
	}
}
\end{lstlisting}

As can be seen in the example, each instruction starts with a (unique) name. It
is then followed by one or more opcodes, which will be explained later on.
Optionally the amount of clock cycles can be overridden, otherwise the default
(as defined in the parameters) is used. This is followed by zero or more
arguments (the local variables) seperated by commas. The instruction body
contains one or more expressions.

\subsection{Opcodes}
Opcodes can be specified as a binary number with optional variables mixed in.
These variables are matches against the actual opcode and read left-to-right. 
The simulator will create a bitmask with all the static numbers and match each
opcode against this bitmask.

A known limitation is that there is no detection of overlapping bitmasks, so be
careful with defining them. In order to please the compiler, one should also
define variables as instruction arguments.

\lstset{caption=Opcode specifation}
\begin{lstlisting}
OPCODE		:	'"' ('0' | '1' | LETTER | ' ')+ '"';
\end{lstlisting}

\lstset{caption=Example opcodes}
\begin{lstlisting}
"0000 0000 0000 0000" // Fixed opcode
"00dd dd11 01rr ddrr" // 6-bit variable d and 4-bit variable r
\end{lstlisting}

Given the above example opcode with variables the binary number 0010 0111 0101
1001 would result in d equal to (binary) 100110 and r equal to 0101.

\subsection{Arguments}
Instruction arguments are the local variables and by default unsigned. The
optional tilde ('\~') indicates it is signed. Instruction arguments pointing to
a register should be prepended the letter 'R'. Their value is taken as the
register offset.

\lstset{caption=Argument specifation}
\begin{lstlisting}
arguments	:	argument (',' argument)*;

argument	:	'~'? identifier;
\end{lstlisting}

\lstset{caption=Example arguments}
\begin{lstlisting}
k		// Single argument
~k		// Single signed argument
Rd		// Single argument pointing to a register
Rd, Rr	// Multiple arguments pointing to registers
\end{lstlisting}

\subsection{Expressions}
Expressions are the most complex part of the language because it is the heart
of each specification.
\lstset{caption=Expressions}
\begin{lstlisting}[name=expressions]
variable	:	'$'? IDENTIFIER // Variable
			|	MAP_TYPE '{' operatorExpr '}' // Typed variable
			|	( IDENTIFIER | MAP_TYPE ) '[' operatorExpr ':' operatorExpr ']' // Multi-register variable
			;

word 		:	variable ('.' (NUMBER | '$'? IDENTIFIER))?
			|	NUMBER
			|	'!' word
			;

operatorExpr:	word (('&' | '|' | '^' | '+' | '-' | '*' | '<<' | '>>') operatorExpr)?;

assignExpr	:	variable '=' operatorExpr;

comparison	:	'==' | '<' | '=<' | '>' | '>=';

condition	:	word comparison word
			|	'(' operatorExpr ')' comparison word
			;

ifExpr		:	'if' condition '{' expr+ '}' ('else' '{' expr+ '}')?;

expr		:	assignExpr ';'
			|	ifExpr
			|	'HALT' ';'
			;
\end{lstlisting}

As can be seen, there are four types of expressions. The halt statement is a
special statement and indicates that the simulator can halt execution. This can
be used instead of infinite jumping or interrupt handling.

\subsubsection{Assignments}
\lstset{caption=Assignments}
\begin{lstlisting}
// assignExpr: write to ram location ram[$SP+1..$SP] the value PC + 1
ram(ram[$SP+1..$SP]) = PC + 1;

// Decrease ram[$SP+1..$SP] by 2
ram[$SP+1..$SP] = ram[$SP+1..$SP] - 2;
\end{lstlisting}

\subsubsection{If statement}
\lstset{caption=If statement}
\begin{lstlisting}
if $k == -1 {
	HALT; // Jump to itself is an infinite loop, so halt
} else {
	// assignExpr
	PC = PC + $k + 1;
}

if (N ^ V) == 0 {
		PC = PC + $k + 1;
} else {
		PC = PC + 1;
}
\end{lstlisting}

\subsubsection{Operator executions}
\lstset{caption=Operator expressions}
\begin{lstlisting}
// Bit 7 from register 31 bitwise and bit 5 register 30
R31.7 & R30.5

// Not bit 7 of register d where variable d is resolved at runtime
!Rd.7
\end{lstlisting}

\subsection{Common tricks}
\subsubsection{Few bit registers in opcodes}
It is very common in some specifications to use a few bits in the opcode to
choose a fixed few registers. For example, the adiw-instruction from the
ATMEL-specification uses two bits to choose one of four registers. This can
easily be fixed as follows. Note that \$d in the refers to the value of d while
Rd refers to the register with an offset of d (thus \$d).

\lstset{caption=Two-bit register variable in opcodes}
\begin{lstlisting}
adiw "1001 0110 KKdd KKKK", clock=2 Rd, K {
	// d = {24,26,28,30}
	$d = 24 + $d * 2;
	// ...
	V = !Rd.7 & $R.15;
	// ...
}
\end{lstlisting}

\subsubsection{Special variable \$R}
The variable \$R by default holds the value of the first expression in an
instruction, but can be overwritten. Again the adiw-instruction from the
ATMEL-specification. In this case we first need to use the few bit register
trick which sets \$R to the wrong value.

\lstset{caption=Special variable \$R}
\begin{lstlisting}
adiw "1001 0110 KKdd KKKK", clock=2 Rd, K {
	// d = {24,26,28,30}
	$d = 24 + $d * 2;
	$R = R[$d+1..$d] + $K;
	// ...
}
\end{lstlisting}
