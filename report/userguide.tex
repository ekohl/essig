\chapter{User guide}
\section{Basics}
\subsection{Style}
The \ac{ESSIG} specification style is influenced by Java. The biggest
difference can be found in typing which we shall see later on. However, both
separate lines by a semi-colon and it can also can be seen in blocks which have
the following pattern:

\begin{verbatim}
IDENTIFIER {
	line1;
	line2;
	...
}
\end{verbatim}


\subsection[IDENTIFIER]{Identifiers}
Identifiers are names and must start with a letter and may be followed with
more letters and digits:

\begin{verbatim}
fragment DIGIT	:	'0'..'9';
fragment LETTER	:	('a'..'z' | 'A'..'Z');
IDENTIFIER	:	LETTER (LETTER | DIGIT)*;
\end{verbatim}

\subsection{Numbers}
Numbers can be specified either as numbers with an optional sign to indicate it
is a negative number or hexadecimal numbers.

\begin{verbatim}
fragment DIGIT	:	'0'..'9';
fragment NUM	:	'-'? DIGIT+;
fragment HEX_LETTER:	('a'..'f'|'A'..'F');
fragment HEX_NUM:	'0' 'x' (DIGIT | HEX_LETTER)+;
NUMBER		:	(NUM | HEX_NUM);
\end{verbatim}

Example:
\begin{verbatim}
10	// Number
-7	// Negative number
0x15f	// Hexadecimal number
\end{verbatim}

\subsection{Comments}
Comments are C-style:

\begin{verbatim}
COMMENT		:	'//' ~('\n'|'\r')* '\r'? '\n'
		|	'/*' .* '*/'
		;
\end{verbatim}

Example:
\begin{verbatim}
// Single line comment
/*
multi
line
comment
*/
\end{verbatim}

\subsection{Opcodes}
\begin{verbatim}
OPCODE		:	'"' ('0' | '1' | LETTER | ' ')+ '"';
\end{verbatim}

Example:
\begin{verbatim}
"0000 0000 0000 0000" // Fixed opcode
"00dd dd11 01rr ddrr" // 4-bit variable d and 4-bit variable r
\end{verbatim}

\section{Specifying a microcontroller}
In order to specify a microcontroller, we have the basic block pattern with an
identifier as a name and must contain four blocks:
\begin{verbatim}
microcontroller	:	IDENTIFIER '{'
				parameters
				registers
				maps
				instructions
			'}'
		;
\end{verbatim}

\subsection{Parameters}
Parameters follow the basic block pattern with a parameter specification on
each line.

\begin{verbatim}
parameters	:	'parameters' '{' (parameter ';')+ '}';

parameter	:	'gprs' NUMBER ('+' NUMBER)?
		|	'opcode-size' NUMBER
		|	'clock' NUMBER
		|	'endianness' ('big' | 'little')
		;
\end{verbatim}

Example:
\begin{verbatim}
parameters {
	gprs 32+5; // 32 general purpose registers at offset 5
	opcode-size 16; // 16 bits opcodes
	clock 1; // Standard 1 clock cycle
	endianness little; // Little endian
}
\end{verbatim}
