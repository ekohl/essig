\chapter{Simulator}
The final simulator is composed of two components: A general VM and a
simulator which is generated. The VM contains things common to all 
microcontrollers and supplies debug functionality. The simulator is generated from a description file. Communication between the VM and the generated simulator is accomplished 
by calling functions directly with the current state of the simulation. The following
subsections describe each component in detail.

\section{VM}

The VM is a library in which functions are gathered that most simulators
have in common. These functions are defined in terms of a private API
for which implementations can be generated using the generator and the
definition language. Linked together they form a library that can run
programs like they were running on the simulated microcontroller. It
is state driven in that most functions are parameterised with a state
which is then manipulated to the desired state by the function. This
design gives the client the freedom to manipulate and read the
state, which when used properly gives interesting possibilities when
debugging a program. The interpreter can keep a list of differences of the
state so program execution can be reversed at any time to a desired point,
after which the user may modify the state manually from the debugger after
which execution can be resumed with a modified state.

\section{Micro Controller state}
To be able to let the VM execute any code for any microcontroller we
defined a structure that could represent any microcontroller state. It
has all things that all microcontrollers have in common, namely a RAM space, a
ROM space, a register file and an I/O space.
Our specification has all information needed
to generate this state and that is exactly what the VM does. It uses the
description compiled into the simulator part and creates an initial state for 
a simulation. This state is
then passed to every function that needs it to operate, which means that the 
VM can for example manage and run several programs simultaneously (although
the debugger frontend does not currently support multiple simultaneous
simulations). \\
In addition to a simulation state, the user may also decide to keep track of
differences in the state, which is managed by a separate stack of diffs.
These diffs can be used to backtrack the program execution. This relatively loose structure gives
a lot of interesting capabilities to the client (e.g. save the diffs
and state to the disk and resume execution later on (with
full backtracking)).

\section{The Generated Simulator}
The generated simulator exposes a private API which contains the
implementations that simulate the instructions, the information 
for the VM to allocate an appropriate simulation state and the information 
to disassemble the target instruction set into a representation that allows 
efficient lookup. The opcode specification is stored in small 
structures, each of which contains a pointer to the associated instruction 
handler, the name of the opcode, the bits that constitute the opcode and a 
bitmask that can be applied to an instruction in order to obtain the 
associated opcode. The
instruction handlers are private functions of the generated simulator,
but they all have the same signature as declared by the VM. During program
execution (i.e., during simulation), the VM passes execution to these handlers
whenever an instruction needs to be executed. These handlers then manipulate
the state through functions exposed by the VM.

\section{The Debugger}
The debugger frontend provides the user with a debugger that can control and
inspect the simulation. Breakpoints may be set for symbol names or addresses,
programs may be simulated (multiple times) and whenever execution is halted
(or even after the program has terminated) the simulation state may be inspected and
altered and execution may be reversed. The frontend uses the readline library
(indirectly) to provide command line editing and history.

\section[Interrupt Handling]{Interrupt Handling}
\begin{comment}
Interrupts in \ac{ESSIG} can occur at two different levels:

\begin{enumerate}
\item {
From outside of the VM through the vm\_interrupt method }
\item {
From a peripheral }
\end{enumerate}
\end{comment}
Interrupts may be specified by callbacks that can be registered with the VM
(with the \texttt{vm\_register\_interrupt\_callable} function). When the debugger
is used, plugins may be written in Python and put into the \texttt{vm/plugins}
directory.\\
Each simulation may have a policy that can govern whether or not the interrupt will be
passed on to the simulator. When interrupts should be handled, a user-supplied
\texttt{set\_interrupt} method will be called at every step with the state of the
simulation and a (potentially NULL) diff pointer. The set\_interrupt function
should then inspect \texttt{state->interrupts} and modify the state of the
microcontroller appropriately. A subsequent \texttt{handle\_interrupt}
function (with a signature equivalent to \texttt{set\_interrupt}) is 
called at every step to check the simulation state for an
interrupt flag. Based on the intricacies of the microcontroller, this function
should then modify
the passed-in state accordingly. This may mean pushing the program counter on the stack,
changing
the stack pointer and writing the address of the interrupt handler specified
in the debuggee to the program counter.

After all this the VM resumes execution with the potentially modified state.
It is up to the implementor of \texttt{set\_interrupt} and 
\texttt{handle\_interrupt} to use the functions of the VM to modify the state.
If those functions are used along with a diff list (a linked list of diffs), changes will be
registered in the diffs. If the state is modified without using any VM
function,
changes will not registered in the diffs and hence reverse-stepping will be oblivious to
any changes incurred by an interrupt handler. This means code could be executed
once with an interrupt handler and once without, and changes in state or
control flow may be observed.

\section[Opcode parsing]{Opcode parsing}
In any simulator at some point opcodes will have to be parsed in order
to determine what code to execute. This means that from the opcodes in 
the program information needs to be extracted to determine what instruction 
needs to be executed. This information depends on the instruction set of 
the microcontroller. 

For example consider the following bit pattern as if it were an opcode for an AVR
microcontroller:

\lstset{caption=An example opcde}
\begin{lstlisting}
0000 1101 0000 0001
\end{lstlisting}

In this case we can identify the instruction based on the first six bits. These
tell us that this opcode represents an add instruction \cite[p.17]{atmelISA}.
The remaining bits tell us that the add works on register 16 and 1. They can take any
value but it would still be an add instruction.

So we are looking for a method that can programmatically determine the instruction
from the opcode. There are a lot of different solutions to this problem 
and we will discuss a few of them here and after that explain the solution we chose.

\subsection[Switch statement]{Switch statement}
A lot of simulators out there use a switch statement to parse opcodes. The idea
is that you input the opcode or a part of the opcode into a switch
statement and continue execution at the matched case. This method has
one disadvantage: The case statement can't be generalized to any
simulator since the opcodes are (or can be) very different accross
different architectures, unless the instructions are disassembled first and the
switch is on the disassembled opcodes (or some other associated unique key) instead.

\lstset{caption=Example of a switch statement for opcode parsing}
\begin{lstlisting}
opcode = readNextOpcode()

switch (opcode & mask)
	case someinstructionpattern:
		dispatchsomeinstruction()
		break;
	case someotherinstructionpattern:
		dispatchsomeotherinstruction()
		break;
	case somegrouppattern:
		switch (opcode & groupmask)
			//More case statements
	...
	default:
		//unrecognized instruction
\end{lstlisting}

\subsection{Some sort of function array}
It isn't used that much at all but for smaller architectures it can be beneficial 
to have a direct mapping of ordered opcodes to functions. The idea is that 
you have an array of functions that handle an
opcode and that you index the array using the opcode. But as mentioned
for architectures with a relatively large opcode this array would
become far too large and since some instructions have some argument(s)
embedded in the opcode a lot of functions appear multiple times in the array.

\lstset{caption=Example opcode parsing using a function array}
\begin{lstlisting}
instructionhandlers = {Long array of functions}

instructionhandlers[readNextOpcode()]()

\end{lstlisting}

\subsection{Our solution}
We wanted to be able to find our instruction in an array efficiently without
having to keep entries for every possible opcode. So instead we list the
opcode with variable parts masked to zero, and a bitmask that can be applied
to an instruction to obtain the associated opcode. These opcodes and masks are
specified in the microcontroller description file. Before execution starts,
all the instructions are disassembled, which means that each instruction holds
a key from which an instruction handler can be obtained in constant time. This gives the advantage of fast opcode resolution in exchange for a slightly longer load time.
\lstset{caption=An example of our opcode parsing solution}
\begin{lstlisting}
instruction_handlers = {
    {handler_name, opcode, instruction_mask, handler},
    ...
}

main_loop(disassembled_code) {
    while true
        fetch pc
        sanity check pc
        Instruction instr = disassembled_code[pc]
        instr.handler(simulation_state, instr.instruction)
}

\end{lstlisting}

It is then up to the individual instruction handlers to parse out any
arguments coded into the instruction.


% TODO: This idea is further illustrated in the following diagram:

\section[PC incrementation]{PC incrementation}
In different microcontrollers the program counter either points to the
current instruction or the next instruction. Because we delegate
program counter incrementation to our instruction handlers, both
situations can be emulated quite easily by either incrementing the
program counter at the beginning of the instruction handler (to
indicate the next instruction) or at the end of the handler.

To illustrate this consider the following two pseudo code
implementations of rcall:

\lstset{caption=PC points to next instruction}
\begin{lstlisting}
PC = PC + 1
PUSH PC
PC = PC + k
\end{lstlisting}

\lstset{caption=PC points to current instruction}
\begin{lstlisting}
PUSH (PC + 1)
PC = PC + k
PC = PC + 1
\end{lstlisting}

It can be seen that in a microcontroller in which the PC points to the
next instruction the push operation is simpler.

In the above example the effects of the instruction was exactly the same
but in microcontrollers in which it is possible to push the PC (This
is not so in atmel microcontrollers) there can be a difference:

\lstset{caption=PC points to next instruction}
\begin{lstlisting}
PC = PC + 1
STORE PC on stack
\end{lstlisting}

\lstset{caption=PC points to current instruction}
\begin{lstlisting}
STORE PC on stack
PC = PC + 1
\end{lstlisting}

Now in the first the top of the stack is the address of the next
instruction but in the second the address of the current instruction.
It also illustrates that our language can handle both situations
because the incrementation of the PC is handled by the instruction
handler.
