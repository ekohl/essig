\chapter{AVR Microcontroller}
For this project, the AVR \ac{UC} is chosen as a base for the \ac{ESSIG}. This
is because the AVR instruction set has simple instructions. For every \ac{UC}
available (e.g. AVR, PIC), there is a specification available. This is often a
PDF file which describes the \ac{UC} in a semi-formal way, which makes it hard
to parse.

Because of this a new specification language had to be created. The goal of
this new language, is to formally describe the entire \ac{UC}s specification
and make it easy to parse. With this specification, one could generate a
simulator. Thus creating a simulator for a \ac{UC} should be as easy as writing
a specification for that microcontroller.
