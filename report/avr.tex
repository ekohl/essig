\chapter{AVR Microcontroller}
For this project, the AVR \UC is chosen as a base for the \ac{ESSIG}. This
is because the AVR instruction set has simple instructions. For every \UC
available (e.g. AVR, PIC), there is a specification available.
 
% Wilfried: Feels like this shouldn't be here but in earlier chapter
%
% Because of this a new specification language had to be created. The goal of
% this new language, is to formally describe the entire \UCs specification
% and make it easy to parse. With this specification, one could generate a
% simulator. Thus creating a simulator for a \UC should be as easy as writing
% a specification for that microcontroller.

\section{AVR Instruction Set}

The AVR instruction set as used by most Atmel AVR microcontrollers has 142 
defined instructions most of which can be found on any AVR microcontroller 
however this does not go for all instructions.

It is a RISC architecture with 32 general purpose registers. 

\subsection{Stack}

The AVR has a dynamic stack that is located in the RAM of the AVR.
The \ac{SP} points past the top of the stack and gets lower if data is 
stored in it.

\subsection{Status Register}

The AVR has a special register in which various results of arithmetic
operations are stored. This \ac{SREG} is composed of siz arithmetic
flags and the global interrupt enable flag and BST and BLD flag are 
stored here as well.

The flags in the \ac{SREG} are:

\begin{description}
\item[I] \hfill \\
The global interrupt enable flag
\item[T] \hfill \\ 
The transfer bit used in BST and BLD instruction % Explain purpose here
\item[H] \hfill \\ 
The half carry flag. It is used to detect unsigned overflow in the 
lower nibble of a register.
\item[S] \hfill \\ 
The signed flag. Indicates if the register is considered signed under 
certain circumstances usually corresponds to N xor V
\item[V] \hfill \\
The signed overflow flag. Set if the preceding arithmetic operation
resulted in an overflow of the signed value of the register.
\item[N] \hfill \\ 
Negative flag. Set if the result is negative
\item[Z] \hfill \\ 
Zero flag. Set if the result was zero
\item[C] \hfill \\ 
The carry flag. Set if unsigned overflow occured during the arithmetic 
operation
\end{description}

It should be noted that not all arithmetic operations affect the flags in the same
way, but the meaning of the flags is generally the same.

\section{ATMega16 - The Model of Choice}

We chose the ATMega16 as the specific microcontroller to use. It is an 
impressive little machine supporting 131 of the AVR instructions
having 16kB of in-system programmable flash memory as well as 1kB of 
internal SRAM. In can run up to 16 MIPS at 16Mhz.

\subsection{Registers}

The ATMega16 features 96 registers (32 general purpose and 64 special 
purpose and I/O registers).

The \ac{SREG} is stored in the I/O space and contains arithmetic flags
and the global interrupt flag.

The \ac{SP} is also stored in the I/O space.

There are various other registers in the I/O space usually corresponding
to some funtion of a peripheral that will be explained in the peripheral
section

\subsection{Peripherals}

The ATMega16 is equipped with a lot of onchip peripherals and also some
generic ports.

A list of peripherals:
\begin{description}
\item{Two 8-bit Timer/Counters with Separate Prescalers and Compare Modes} 
\item{One 16-bit Timer/Counter with Separate Prescaler, Compare Mode, and Capture Mode}
\item{Real Time Counter with Separate Oscillator}
\item{Four PWM Channels}
\item{8-channel, 10-bit ADC}
\item{Byte-oriented Two-wire Serial Interface}
\item{Programmable Serial USART}
\item{Master/Slave SPI Serial Interface}
\item{Programmable Watchdog Timer with Separate On-chip Oscillator}
\item{On-chip Analog Comparator}
\end{description}

\subsection{Interrupts}

Though interrupt procedures differ widely amongst microcontrollers (and processors)
it is important to understand the mechanics of it on the ATMega16. 

There are two types of interrupts. Simple interrupts that are caused by a set 
interrupt flag and interrupts that are only work if a certain condition is met 
(if this condition appears but disappears before the condition is checked nothing 
will happen). These do not necessarily set an interrupt flag.

When an interrupt occurs the global interrupt flag is cleared disabling all 
interrupts. The current program counter is stored on the stack and the 
program counter is set to the interrupt vector corresponding to the specific 
interrupt. The lower the interrupt vector the higher the interrupt's priority is

When the microcontroller exits from the interrupt it allways first returns to the
main program to execute at least one instruction before checking on any other
interrupts.
