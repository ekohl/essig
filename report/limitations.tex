\chapter{Limitations and Future Work}

\section{Microcontroller Semantics and Specification Language}
Currently there is no support for floating point arithmetic. This could be
added in future work.\\
\\
Currently it is not possible to specify semantics of I/O ports, which may be a
welcome addition for users.\\
\\
For improved debugging support it may be necessary to specify the calling
convention.

\section{VM Improvements}
The VM could be improved by including support for more types of executables, as
it currently only supports loading ELF executables. It is however unclear if there are
microcontroller architectures for which no compiler is available that can
generate ELF executables.\\
\\
The ELF loader could be improved with support for executables with multiple 
executable segments, although it is unclear if any compilers for microcontrollers
generate such executables.

\section{Debugger Frontend Improvements}
Currently it is not possible to specify the interrupt policy from the
debugger, nor is it possible to specify interrupts. This functionality can
only be accessed through custom user code, although there is no good reason
for this.\\
\\
Another welcome additions may include symbolic debugging (other than
breakpoints). It's currently not possible to inspect local variables,
arguments to functions, etc. It's also not possible to print a backtrace, or
to execute arbitrary debuggee code from the debugger command line. 
However, implementing all these
features may be a tremendous amount of work and time may be better spent on
generating a simulator that can be plugged into an existing debugger, such as
GDB.

\section{Optimizations}
Opcode arguments are currently unpacked from the instruction in the opcode
handler, which means extra runtime overhead that could be prevented by doing
the unpacking in the disassembler instead. This could be implemented by e.g.
listing additionaly all the masks that could parse out arguments of
instructions in the generated simulator, and then subsequently parsing these
out in the disassembler, storing them on the heap in a chunk the size of 
OPCODE\_TYPE * the number of arguments, and passing the instruction handler a 
pointer to the chunk.
