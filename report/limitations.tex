\chapter{Limitations and Future Work}

\section{Microcontroller Semantics and Specification Language}
Improvement of the status registers should be high on the priority list.
Currently these registers are hardcoded in the code generator which is bad. It
does bring a big performance benefit, but this should be changeable in the
specification. One can think of a complex register specification which allows
mapping multiple status bits into a single register and is only read once and
written once per instruction.\\
\\
Since microcontrollers often have no support for floating point operations,
there is currently no support for floating point arithmetic. This could be
added in future work.\\
\\
Another limitation is the limited support of I/O ports. While it is possible to
map a block of memory and read this, it is not possible to specify the
semantics of I/O ports, which may be a welcome addition for users.\\
\\
For improved debugging support it may be necessary to specify the calling
convention.\\
\\
A possible improvement for opcodes is rewriting the Java-based parsing with
ANTLR. It could also be extended with some syntax to choose the byte order for
variables.\\
\\
Furthermore, the typing could be improved. For registers, the R is part of the
variable name and stripped in the actual use while there is also the special
variable R which contains the result. Then there's MAP\_TYPE(...) for ram, rom,
etc. Since registers are internally no different than all other memory
mappings, it would be cleaner and more consistent if the stripping would be
removed and R31 would be changed into R(31).\\
\\
The current state of the checker is another big improvement that would greatly
enhance the user experience. Presently it's very limited and it could warn the
user much earlier on. As noted in the user guide, one could check for
overlapping opcode bitmasks.

\section{VM Improvements}
The VM could be improved by including support for more types of executables, as
it currently only supports loading ELF executables. It is however unclear if there are
microcontroller architectures for which no compiler is available that can
generate ELF executables.\\
\\
The ELF loader could be improved with support for executables with multiple 
executable segments, although it is unclear if any compilers for microcontrollers
generate such executables.

\section{Debugger Frontend Improvements}
Currently it is not possible to specify the interrupt policy from the
debugger, nor is it possible to specify interrupts. This functionality can
only be accessed through custom user code, although there is no good reason
for this.\\
\\
Another welcome additions may include symbolic debugging (other than
breakpoints). It's currently not possible to inspect local variables,
arguments to functions, etc. It's also not possible to print a backtrace, or
to execute arbitrary debuggee code from the debugger command line. 
However, implementing all these
features may be a tremendous amount of work and time may be better spent on
generating a simulator that can be plugged into an existing debugger, such as
GDB.

\section{Optimizations}
Opcode arguments are currently unpacked from the instruction in the opcode
handler, which means extra runtime overhead that could be prevented by doing
the unpacking in the disassembler instead. This could be implemented by e.g.
listing additionally all the masks that could parse out arguments of
instructions in the generated simulator, and then subsequently parsing these
out in the disassembler, storing them on the heap in a chunk the size of
OPCODE\_TYPE * the number of arguments, and passing the instruction handler a
pointer to the chunk.

Another optimization would be the addition of a JIT compiler, that would
JIT-compile hot blocks of code which are often executed and translate the
target instructions to host instructions directly \cite{JIT}.

\section{Build system}
The build system is, as documented in the README, somewhat hardcoded towards
using atmel specification file. It also fails to correctly clean generated
simulator code and requires git to do so.

Besides that, it's not very user-friendly. This could be improved by adding
checks if all libraries are all present.
