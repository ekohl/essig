\chapter{Grammar}

The grammar consists of 4 sections. These sections are:
\begin{itemize}
\item Parameters
\item Registers
\item Maps
\item Instructions
\end{itemize}

\subsection{Parameters}
In this section al parameters of a \ac{UC} are specified. Parameters are: 

\begin{itemize}
\item number of \ac{GPRS} 
\item endianness
\item standard clock cycles needed for 1 instruction (this can be overrided by an instruction if an instruction needs more than 1 cycle)
\item opcode size (AVR uses 16 bit opcodes, PIC uses 14 bit opcodes).
\end{itemize} 

\subsection{Registers}
In this section the offsets of register values are specified. Currently a little hack is used to address bits in the \ac{SREG}. (TODO suply example)

\subsection{Maps}
A \ac{UC} has several ranges of addresses. These can be specified below. These address ranges are:
\begin{itemize}
\item chunk = The whole block of memory used for the microcontroller (all other ranges need be addressed in this chunck).
\item register = The location of the \ac{GPRS}.
\item io = The input and output ports of the microcontroller.
\item ram = The \ac{RAM} address range, this is memory that can be used by the userprogram.
\item rom = The \ac{ROM} address range, the user code is stored in here. 
\item print = This is a custom range used for outputting information in the command line information of the simulator. If in the simulator a value is written to an address in this range, the simulator wil print this value.
\end{itemize}

\subsection{Instructions}
The biggest part of the grammar, the instructions. Each instruction needs an opcode. For usage of these instructions, check the userguide.

\subsection{Missing functionality}
TODO