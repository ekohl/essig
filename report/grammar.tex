\chapter{Grammar}

\section{The Grammar}
The grammar consists of 4 sections. These sections are:
\begin{enumerate}
\item Parameters
\item Registers
\item Maps
\item Instructions
\end{enumerate}

\subsection{Parameters}
In this section all parameters of a \UC are specified. Parameters are: 


\begin{itemize}
\item number of \ac{GPRS} 
\item endianness
\item standard clock cycles needed for 1 instruction (can be overridden per instruction)
\item opcode size (AVR uses 16 bit opcodes, PIC uses 14 bit opcodes).
\end{itemize} 

\subsection{Registers}
In this section the offsets of register values are specified. Currently a little hack is used to address bits in the \ac{SREG}.

\subsection{Maps}
A \ac{UC} has several ranges of addresses. These can be specified below. These address ranges are:
\begin{itemize}
\item chunk = The whole block of memory used for the microcontroller (all other ranges need be addressed in this chunk).
\item register = The location of the \ac{GPRS}.
\item io = The input and output ports of the microcontroller.
\item ram = The \ac{RAM} address range, this is memory that can be used by the user program.
\item rom = The \ac{ROM} address range, the user code is stored in here. 
\item print = This is a custom range used for outputting information in the command line information of the simulator. If in the simulator a value is written to an address in this range, the simulator will print this value.
\end{itemize}

\subsection{Instructions}
The biggest part of the grammar, the instructions. Each instruction needs an opcode. For usage of these instructions, check the user guide.

\section{Design choices and problems}
The biggest problem encountered was that we didn't have a good look at the overall \UC specification. We just started to work on some instructions and adjusted the grammer accordingly, if we would have overviewed all (or a large subset of) the instructions, we would have saved some time during the development of the grammar. We would have foreseen problems like endianess / signed and unsigned / bits and multiword registers.

\subsection{Opcodes}
We chose to parse the opcodes, not in the ANTLR language, but in inline Java code. This is done because we found it much easier to do in Java.

\section{Future of the grammar}
Things that have to be done to complete the grammar are:
\begin{itemize}
\item Check compatibility of the grammar with other {\UC}'s than the ATMEL AVR.
\item Add support for floating point numbers.
\item In the section 'parameters' the size of a register should be added, e.g. "register-size = 8;" which are 8 bits.
\item The section 'registers'
\begin{itemize}
\item This section should be renamed to a better suitable name such as 'offsets'.
\item Declaration of single BIT's should be possible. The bits in the \ac{SREG} are now set with a little java hack in the code, it checks if the IDENTIFIER is one of these "CZNVSHTI".  
\item Size of an offset should be declareable.
\end{itemize}
\item Declaration of \ac{UC} peripherals like USART, SPI, I2C, ADC.
\end{itemize}
