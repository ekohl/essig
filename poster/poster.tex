\documentclass{beamer}
\mode<presentation>
  {
%   \usetheme{Berlin}
  %\usetheme{Dreuw}
  \usetheme{Dreuw}
  }
%  \usepackage{times}
  \usepackage{subfigure}
  \usepackage{relsize}
  \usepackage{tikz}
  \usepackage{listings}
  \usepackage{wrapfig}
  \usepackage{verbatim} 
  \usepackage{amsmath,amsthm, amssymb, latexsym}
 % \boldmath
  \renewcommand*{\thesubfigure}{}
%  \renewcommand{\figurename}{bvla}

%\usepackage{picins}
%\newcommand{\infobox}[2]{
%    \parpic(0.34\textwidth,0pt)[lf]{
%        \parbox[b]{0.32\textwidth}{
%             \bigskip {\bf #1}  \small{{{\sffamily #2}}} \bigskip
%        }
%    }
%    \bigskip
%}

\makeatletter
\providecommand\dotsum{\mathpalette\@dotted\sum \vphantom{\sum}}
\def\@dotted#1#2{\ooalign{\hfil$#1 \bullet $\hfil\cr\hfil$#1#2$\hfil\cr}}
\makeatother
\DeclareMathOperator*{\psum}{\mathchoice{\sum\!\!\!\!\!\!\!\bullet\ \ }{\sum\hspace{-2.2ex}\raisebox{0.2ex}{\scalebox{0.75}{$\bullet$}}\,\,}{\sum\!\!\!\!\!\!\!\bullet}{\sum\!\!\!\!\!\!\!\bullet}}
\DeclareMathOperator*{\psumm}{\mathchoice{\sum\!\!\!\!\!\!\!\bullet\ \ }{\sum\hspace{-2.6ex}\raisebox{0.15ex}{\scalebox{0.75}{$\bullet$}}\,\,}{\sum\!\!\!\!\!\!\!\bullet}{\sum\!\!\!\!\!\!\!\bullet}}

\DeclareMathOperator*{\pplus}{\oplus}
\newcommand{\dv}{\vec{t}}

  \usetikzlibrary{backgrounds}
\usetikzlibrary{automata}
\usetikzlibrary{arrows}
\usetikzlibrary{shapes}
\usetikzlibrary{decorations.pathmorphing}
\usetikzlibrary{fit}

\let\picturesize\Large

\def\clap#1{\hbox to 0pt{\hss#1\hss}}
\def\mathclap{\mathpalette\mathclapinternal}
\def\mathllap{\mathpalette\mathllapinternal}
\def\mathrlap{\mathpalette\mathrlapinternal}
\def\mathclapinternal#1#2{%
            \clap{$\mathsurround=0pt#1{#2}$}}
\def\mathllapinternal#1#2{%
            \llap{$\mathsurround=0pt#1{#2}$}}
\def\mathrlapinternal#1#2{%
            \rlap{$\mathsurround=0pt#1{#2}$}}


%%%%% TIKZ
\tikzstyle{every picture}=[line width=4pt]
\tikzstyle{ultra ultra thick}=[line width=2.5pt]
\tikzstyle{nodeSmall} = [state, minimum size=6mm, inner sep=0mm, fill=black, node distance=1.25cm]
\tikzstyle{state2}=[state, minimum size=7mm, inner sep=0.5mm]
\tikzstyle{teststate}=[state, minimum size=4mm, inner sep=0.5mm]
\tikzstyle{testleaf}=[minimum size=4mm, inner sep=0.5mm]
\tikzstyle{every loop}=[->]
\tikzstyle{onEdge}=[fill=white, pos=0.4]
\tikzstyle{testschuin}=[node distance=\testschuin]
\tikzstyle{test}=[node distance=\test, pos=0.4]
\tikzstyle{loopupper}=[in=67, out=113, loop]
\tikzstyle{lagerLabel}=[pos=0.68]
\tikzstyle{every scope}=[>=latex, node distance=\recht]
\tikzstyle{smallstate}=[state, minimum size=3mm, inner sep=0.5mm, fill=white, ultra ultra thick]
\tikzstyle{smallstate3}=[state, minimum size=3mm, inner sep=2mm, fill=white]
\tikzstyle{smallrectangle}=[state, rectangle, minimum size=3mm, inner sep=0.5cm, fill=white]
\tikzstyle{smallstate2}=[state, minimum size=5mm, inner sep=0.5mm, fill=white]
\tikzstyle{background rectangle}= [rounded corners, fill=yellow!20, draw=black, rounded corners=1ex]
\tikzstyle{oval} = [state, ellipse, minimum size=4mm, inner sep=0.5mm, node distance=1.5cm]
\tikzstyle{prob}=[->, decorate, decoration={snake,segment length=2mm, amplitude=.4mm,post length=1mm}]




\tikzstyle{loopleft}=[in=-176, out=-184, loop, swap]
\definecolor{greenish}{rgb}{0,0.5,0}
\tikzstyle{loopright}=[in=-10, out=10, loop, swap]
\tikzstyle{loopleft2}=[in=-171, out=-189, loop, swap]

\newlength{\testschuin}
\newlength{\test}
\newlength{\recht}
\setlength{\testschuin}{2.1213203435596425732025330863145cm}
\setlength{\test}{1.5cm}
\setlength{\recht}{1.7677669529663688110021109052621cm}
\newcommand{\back}{\picturesize \color{yellow!20}.}
\newcommand{\fail}{\bf \vphantom{p}fail\vphantom{f}}
\newcommand{\pass}{\bf \vphantom{p}pass\vphantom{f}}
%%%%%%%% TIKZ


  \usepackage[english]{babel}
  \usepackage[latin1]{inputenc}
%   \usepackage[textpos}
  \usepackage[absolute,overlay]{textpos}
  \usepackage[orientation=portrait,size=a0,scale=1.35,debug]{beamerposter}

  %%%%%%%%%%%%%%%%%%%%%%%%%%%%%%%%%%%%%%%%%%%%%%%%%%%%%%%%%%%%%%%%%%%%%%%%%%%%%%%%%5
  \graphicspath{{figures/}}

  \title[]{\LARGE \textbf{Embedded System SImulator Generator}}
  \author[]{\large
	Mark Florisson,
	Ruud Harmsen,
	Wilfried van Asten,
	Ewoud Kohl van Wijngaarden
  }

  \institute{\Large \vfill {Ontwerpproject (192199109) -- Faculty of Electrical Engineering -- University of Twente} }
  \date{}

  \begin{document}
  \begin{frame}{}

\vspace{-1cm}
% Eerste rij
\begin{columns}[t]
\begin{column}{.02\linewidth}\end{column}
\begin{column}{.97\linewidth}\begin{block}{\large \smash{Introduction}\vphantom{Introduction}}
\begin{columns}[T]
\begin{column}{0.01\linewidth}\end{column}
\begin{column}{.48\linewidth}
Writing microcontroller simulators is repetative and error prone work. We automate this into:
\begin{enumerate}
  \item Write specification
  \item Compile into VM
  \item Compile VM with generic simulator code
  \item Profit
\end{enumerate}
\end{column}
\begin{column}{.02\linewidth}
\begin{tabular}{cc|}
&\\
&\\
&\\
&\\
&\\
&\\
&\\
&\\
&\\
&\\
&\\
\end{tabular}
\end{column}
\begin{column}{.48\linewidth}
\begin{center}
	\includegraphics{figures/design_diagram.png}
\end{center}
\end{column}
\begin{column}{0.01\linewidth}\end{column}
\end{columns}
\end{block}
\end{column}
\begin{column}{.02\linewidth}\end{column}
\end{columns}

\vskip0.01\linewidth
%\vfill

% Tweede rij
\begin{columns}[t]
\begin{column}{.02\linewidth}\end{column}
\begin{column}{.313\linewidth}\setbeamertemplate{block begin}{
  \vskip.75ex
  \begin{beamercolorbox}[rounded=true,shadow=true,leftskip=1cm,colsep*=.75ex]{block title}%
    \usebeamerfont*{block title}\insertblocktitle
  \end{beamercolorbox}%
  {\ifbeamercolorempty[bg]{block body}{}{\nointerlineskip\vskip-0.5pt}}%
  \usebeamerfont{block body}%
  \begin{beamercolorbox}[ht=27.17cm, rounded=true,shadow=true,colsep*=.75ex,sep=.75ex, vmode]{block body}%
    \ifbeamercolorempty[bg]{block body}{\vskip-.25ex}{\vskip-.75ex}\vbox{}%
    \begin{adjustwidth}{0.8cm}{0.4cm}
  }
  \setbeamertemplate{block end}{
  \end{adjustwidth}
  \end{beamercolorbox}
}

\begin{block}{\large \smash{6. Linearisation}\vphantom{Introduction}}
Given the following specification in prCRL:

\[\ \ \ \ \ \ X = \tau \!\!\psum_{i : \{1,2\}} \frac{i}{3} \colon \Big(\text{send}(i) \cdot X + \sum_{j:\mathbb{N}} j < 10 \Rightarrow \text{send}(j^i)\cdot X \Big)\]

\begin{columns}
\begin{column}{0.02\linewidth}\end{column}
\begin{column}{.55\linewidth}

\noindent \!\!\!\!The corresponding linear form is:
\begin{align*}
&X({\color{blue!70!black}\text{pc} : \{1,2,3\}, i:\{1,2\}}) = \\
& \quad \phantom{{}+{}} \hphantom{{}\sum_{j : \mathbb{N}}{}}{\color{blue!70!black}\text{pc} = 1} \Rightarrow \textstyle \tau \psum_{i:\{1,2\}} \frac{i}{3} \colon X({\color{blue!70!black}2, i})\\
& \quad {}+{} \hphantom{{}\sum_{j : \mathbb{N}}{}}{\color{blue!70!black}\text{pc} = 2} \Rightarrow \text{send}(i) \cdot X({\color{blue!70!black}1}, i)\\
& \quad {}+{} \sum_{j : \mathbb{N}} {\color{blue!70!black}\text{pc} = 2} \wedge j < 10 \Rightarrow \text{send}(j^i) \cdot X({\color{blue!70!black}1}, i)
\end{align*}

\end{column}


\begin{column}{.35\linewidth}
\begin{figure}
\subfigure[A graphical representation of $X$]{

\begin{tikzpicture}[scale=1.25, transform shape, show background rectangle, descr/.style={fill=white,inner sep=2.5pt}]
	\node[smallstate3] (s_0) {\footnotesize $1$};
	\node[smallstate3] (s_1) [right of=s_0, node distance=7cm] {\footnotesize $2$};
	\draw[->] (s_0) edge [bend left=30, auto] node {\footnotesize$\tau \psum_{1:\{1,2\}}\frac{i}{3}$} (s_1);
	\draw[->] (s_1) edge [bend left=0] node [fill=yellow!20] {\phantom{......}} node {\footnotesize$\text{send}(i)$} (s_0);
	\draw[->] (s_1) edge [bend left=30, auto] node {\footnotesize$\sum_{j:\mathbb{N}} j < 10 \Rightarrow \text{send}(j^i)$} (s_0);
	
\end{tikzpicture}
}
\end{figure}
\end{column}
\begin{column}{0.02\linewidth}\end{column}
\end{columns}

\vskip36pt 

For more complicated systems the ideas behind linearisation remain the same:
\begin{adjustwidth}{0.5cm}{0.5cm}
\begin{itemize}
\item \ Introduce a \alert{program counter} to remember the location in the formula
\item \ Introduce \alert{global parameters} to remember bound variables
\end{itemize}
\end{adjustwidth}
\vskip10pt
We developed an \alert{algorithm} to transform any prCRL specification to an LPPE, \alert{proved it correct}, and \alert{implemented} it.



%Given the following specification in prCRL:

%\begin{columns}
%\begin{column}{.5\linewidth}
%\[\ \ \ \ \ \ X = \sum_{d : D} send(d) \cdot \tau \cdot store(d) \cdot X\]
%\end{column}
%\begin{column}{.5\linewidth}
%\begin{figure}
%\subfigure[A graphical representation of the specification $X$]{
%\begin{tikzpicture}[scale=1.25, transform shape, show background rectangle, descr/.style={fill=white,inner sep=2.5pt}, auto]
%	\node[smallstate3] (s_0) {\footnotesize $1$};
%	\node[smallstate3] (s_1) [right of=s_0, node distance=5cm] {\footnotesize $2$};
%	\node[smallstate3] (s_2) [right of=s_1, node distance=5cm] {\footnotesize $3$};
%	\draw[->] (s_0) edge [bend left=15] node {\footnotesize$\sum_{d:D}\text{send}(d)$} (s_1);
%	\draw[->] (s_1) edge [bend left=15] node {\footnotesize$\tau$} (s_2);
%	\draw[->] (s_2) edge [bend left=15] node {\footnotesize$\text{store}(d)$} (s_0);
%	
%\end{tikzpicture}
%}
%\end{figure}
%\end{column}
%\end{columns}

%The corresponding linear form is:
%\begin{align*}
%&X({\color{blue}\text{pc} : \{1,2,3\}}, d:D) = \\
%& \quad \phantom{{}+{}} {\color{blue}\text{pc} = 1} \Rightarrow \textstyle\sum_{d:D} send(d) \cdot X({\color{blue}2, d})\\
%& \quad {}+{} {\color{blue}\text{pc} = 2} \Rightarrow \tau \cdot X({\color{blue}3}, d)\\
%& \quad {}+{} {\color{blue}\text{pc} = 3} \Rightarrow store(d) \cdot X({\color{blue}1}, d)
%\end{align*}
%For more complicated systems the ideas behind linearisation remain the same:
%\begin{adjustwidth}{0.5cm}{0.5cm}
%\begin{itemize}
%\item \ Introduce a \alert{program counter} to remember the location in the formula
%\item \ Introduce \alert{global parameters} to remember bound variables
%\end{itemize}
%\end{adjustwidth}
%\vskip10pt
%An \alert{algorithm} transforming any prCRL specification to an LPPE has been developed, \alert{proven correct}, and \alert{implemented}.
\end{block}\end{column}
\begin{column}{.01\linewidth}\end{column}
\begin{column}{.313\linewidth}\begin{block}{\large \smash{Debugger}\vphantom{Debugger}}

\newcommand{\userInput}[1]{
	\textcolor{red}{#1}  
}

\tiny {

\lstset{
backgroundcolor=\color{black},
basicstyle=\color{white}
}

Run of a simple factorial program: (red text = user input)
\begin{lstlisting}[escapechar=!]
Welcome to the simulator's debugging interface. Type help for help. 
(sim) !\userInput{run}!
Value 0x1c written to 0x3b.
Value 0x8c written to 0x3b.
Value 0xfc written to 0x3b.
Value 0x0 written to 0x3b.
The program has stopped running.
\end{lstlisting}

Reverse stepping(undoing previous step).
\begin{lstlisting}[escapechar=!]
(sim) !\userInput{rstep}!
       rjmp 0xcfff pc: 0xd2 address: 0x1a4
(sim) !\userInput{rstep}!
       cli 0x94f8 pc: 0xd1 address: 0x1a2
(sim) !\userInput{rstep}!
       jmp 0x940c pc: 0x45 address: 0x8a
\end{lstlisting}
Forward stepping, get clock-cycles and get the value of the ProgramCounter register.
\begin{lstlisting}[escapechar=!]
(sim) !\userInput{step}!
       cli 0x94f8 pc: 0xd1 address: 0x1a2
(sim) !\userInput{step}!
       rjmp 0xcfff pc: 0xd2 address: 0x1a4
(sim) !\userInput{info cycles}!
931 cycles have passed.
(sim) !\userInput{info register PC}!
PC              0xd2 address: 0x1a4
\end{lstlisting}

Show disassembled code.
\begin{lstlisting}[escapechar=!]
(sim) !\userInput{disassemble}!
Address        PC    Opcode          Instruction    
     0          0     jmp             0x940c
     2          1     noop            0x002a
   15e         af     pop             0x91cf
   160         b0     pop             0x91df
   162         b1     ret             0x9508
   164         b2     mul             0x9f62
   166         b3     movw            0x01d0
   168         b4     mul             0x9f73
   16a         b5     movw            0x01f0
   16c         b6     mul             0x9f82
   16e         b7     add             0x0de0
   170         b8     adc             0x1df1
   172         b9     mul             0x9f64
   174         ba     add             0x0de0
   176         bb     adc             0x1df1
   178         bc     mul             0x9f92
   17a         bd     add             0x0df0
\end{lstlisting}

}


\end{block}

\end{column}
\begin{column}{.01\linewidth}\end{column}
\begin{column}{.313\linewidth}\input{futurework}\end{column}
\begin{column}{.02\linewidth}\end{column}
\end{columns}

%\vfill
\vskip0.01\linewidth

% Derde rij
\begin{columns}[t]
\begin{column}{.02\linewidth}\end{column}
\begin{column}{.97\linewidth}\input{futureuse}\end{column}
\begin{column}{.02\linewidth}\end{column}
\end{columns}

%\vfill

\end{frame}
\end{document}
