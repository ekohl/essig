\setbeamertemplate{block begin}{
  \vskip.75ex
  \begin{beamercolorbox}[rounded=true,shadow=true,leftskip=1cm,colsep*=.75ex]{block title}%
    \usebeamerfont*{block title}\insertblocktitle
  \end{beamercolorbox}%
  {\ifbeamercolorempty[bg]{block body}{}{\nointerlineskip\vskip-0.5pt}}%
  \usebeamerfont{block body}%
  \begin{beamercolorbox}[ht=33cm, rounded=true,shadow=true,colsep*=.75ex,sep=.75ex, vmode]{block body}%
    \ifbeamercolorempty[bg]{block body}{\vskip-.25ex}{\vskip-.75ex}\vbox{}%
    \begin{adjustwidth}{0.8cm}{0.4cm}
  }
  \setbeamertemplate{block end}{
  \end{adjustwidth}
  \end{beamercolorbox}
}

\begin{block}{\large \smash{Example Specficication}\vphantom{Example Code}}

\tiny {  
\lstset{
numbers=left,                   % where to put the line-numbers
numberstyle=\tiny,      % the size of the fonts that are used for the line-numbers
stepnumber=1,
xleftmargin=25pt
}

'demo' is the type of uC. Could be anything like atmega16. \\
uC = Microcontroller
\begin{lstlisting}
demo {
	...Rest of code below..
}
\end{lstlisting}
Parameters specific for the uC.
\begin{lstlisting}
	parameters {
		gprs 2+5;
		opcode-size 16;
		clock 1;
		endianness little;
	}
\end{lstlisting}
Setup offsets of registers in the uC.
\begin{lstlisting}
	registers {
		SREG	= 0x5F;
		PC	= 0x462;
		SP	= 0x5D;
	}
\end{lstlisting}
The memory mappings of the uC.
\begin{lstlisting}
	maps {
		chunk		(0, 0xFFFFFFF);
		register	(0, 0x20);
		io		(0x20, 0x60);
		ram		(0, 0x461);
		rom		(0x464, 0xFFFFFF);
		print		(0x3b, 0x3c);
	}

\end{lstlisting}
All the instructions of the uC. (this is an example so only 3 are shown).
\begin{lstlisting}
	instructions {
		noop "0000 0000 0000 0000" {
			PC = PC + 1;
		}

		/* Load an I/O Location to Register */
		in "1011 0AAd dddd AAAA" Rd, A {
			Rd = io($A);

			PC = PC + 1;
		}

		jmp "1001 010k kkkk 110k", "kkkk kkkk kkkk kkkk" k {
			PC = $k;
		}
	}
  
\end{lstlisting}

}

\end{block}
