\setbeamertemplate{block begin}{
  \vskip.75ex
  \begin{beamercolorbox}[rounded=true,shadow=true,leftskip=1cm,colsep*=.75ex]{block title}%
    \usebeamerfont*{block title}\insertblocktitle
  \end{beamercolorbox}%
  {\ifbeamercolorempty[bg]{block body}{}{\nointerlineskip\vskip-0.5pt}}%
  \usebeamerfont{block body}%
  \begin{beamercolorbox}[ht=27.17cm, rounded=true,shadow=true,colsep*=.75ex,sep=.75ex, vmode]{block body}%
    \ifbeamercolorempty[bg]{block body}{\vskip-.25ex}{\vskip-.75ex}\vbox{}%
    \begin{adjustwidth}{0.8cm}{0.4cm}
  }
  \setbeamertemplate{block end}{
  \end{adjustwidth}
  \end{beamercolorbox}
}

\begin{block}{\large \smash{6. Linearisation}\vphantom{Introduction}}
Given the following specification in prCRL:

\[\ \ \ \ \ \ X = \tau \!\!\psum_{i : \{1,2\}} \frac{i}{3} \colon \Big(\text{send}(i) \cdot X + \sum_{j:\mathbb{N}} j < 10 \Rightarrow \text{send}(j^i)\cdot X \Big)\]

\begin{columns}
\begin{column}{0.02\linewidth}\end{column}
\begin{column}{.55\linewidth}

\noindent \!\!\!\!The corresponding linear form is:
\begin{align*}
&X({\color{blue!70!black}\text{pc} : \{1,2,3\}, i:\{1,2\}}) = \\
& \quad \phantom{{}+{}} \hphantom{{}\sum_{j : \mathbb{N}}{}}{\color{blue!70!black}\text{pc} = 1} \Rightarrow \textstyle \tau \psum_{i:\{1,2\}} \frac{i}{3} \colon X({\color{blue!70!black}2, i})\\
& \quad {}+{} \hphantom{{}\sum_{j : \mathbb{N}}{}}{\color{blue!70!black}\text{pc} = 2} \Rightarrow \text{send}(i) \cdot X({\color{blue!70!black}1}, i)\\
& \quad {}+{} \sum_{j : \mathbb{N}} {\color{blue!70!black}\text{pc} = 2} \wedge j < 10 \Rightarrow \text{send}(j^i) \cdot X({\color{blue!70!black}1}, i)
\end{align*}

\end{column}


\begin{column}{.35\linewidth}
\begin{figure}
\subfigure[A graphical representation of $X$]{

\begin{tikzpicture}[scale=1.25, transform shape, show background rectangle, descr/.style={fill=white,inner sep=2.5pt}]
	\node[smallstate3] (s_0) {\footnotesize $1$};
	\node[smallstate3] (s_1) [right of=s_0, node distance=7cm] {\footnotesize $2$};
	\draw[->] (s_0) edge [bend left=30, auto] node {\footnotesize$\tau \psum_{1:\{1,2\}}\frac{i}{3}$} (s_1);
	\draw[->] (s_1) edge [bend left=0] node [fill=yellow!20] {\phantom{......}} node {\footnotesize$\text{send}(i)$} (s_0);
	\draw[->] (s_1) edge [bend left=30, auto] node {\footnotesize$\sum_{j:\mathbb{N}} j < 10 \Rightarrow \text{send}(j^i)$} (s_0);
	
\end{tikzpicture}
}
\end{figure}
\end{column}
\begin{column}{0.02\linewidth}\end{column}
\end{columns}

\vskip36pt 

For more complicated systems the ideas behind linearisation remain the same:
\begin{adjustwidth}{0.5cm}{0.5cm}
\begin{itemize}
\item \ Introduce a \alert{program counter} to remember the location in the formula
\item \ Introduce \alert{global parameters} to remember bound variables
\end{itemize}
\end{adjustwidth}
\vskip10pt
We developed an \alert{algorithm} to transform any prCRL specification to an LPPE, \alert{proved it correct}, and \alert{implemented} it.



%Given the following specification in prCRL:

%\begin{columns}
%\begin{column}{.5\linewidth}
%\[\ \ \ \ \ \ X = \sum_{d : D} send(d) \cdot \tau \cdot store(d) \cdot X\]
%\end{column}
%\begin{column}{.5\linewidth}
%\begin{figure}
%\subfigure[A graphical representation of the specification $X$]{
%\begin{tikzpicture}[scale=1.25, transform shape, show background rectangle, descr/.style={fill=white,inner sep=2.5pt}, auto]
%	\node[smallstate3] (s_0) {\footnotesize $1$};
%	\node[smallstate3] (s_1) [right of=s_0, node distance=5cm] {\footnotesize $2$};
%	\node[smallstate3] (s_2) [right of=s_1, node distance=5cm] {\footnotesize $3$};
%	\draw[->] (s_0) edge [bend left=15] node {\footnotesize$\sum_{d:D}\text{send}(d)$} (s_1);
%	\draw[->] (s_1) edge [bend left=15] node {\footnotesize$\tau$} (s_2);
%	\draw[->] (s_2) edge [bend left=15] node {\footnotesize$\text{store}(d)$} (s_0);
%	
%\end{tikzpicture}
%}
%\end{figure}
%\end{column}
%\end{columns}

%The corresponding linear form is:
%\begin{align*}
%&X({\color{blue}\text{pc} : \{1,2,3\}}, d:D) = \\
%& \quad \phantom{{}+{}} {\color{blue}\text{pc} = 1} \Rightarrow \textstyle\sum_{d:D} send(d) \cdot X({\color{blue}2, d})\\
%& \quad {}+{} {\color{blue}\text{pc} = 2} \Rightarrow \tau \cdot X({\color{blue}3}, d)\\
%& \quad {}+{} {\color{blue}\text{pc} = 3} \Rightarrow store(d) \cdot X({\color{blue}1}, d)
%\end{align*}
%For more complicated systems the ideas behind linearisation remain the same:
%\begin{adjustwidth}{0.5cm}{0.5cm}
%\begin{itemize}
%\item \ Introduce a \alert{program counter} to remember the location in the formula
%\item \ Introduce \alert{global parameters} to remember bound variables
%\end{itemize}
%\end{adjustwidth}
%\vskip10pt
%An \alert{algorithm} transforming any prCRL specification to an LPPE has been developed, \alert{proven correct}, and \alert{implemented}.
\end{block}