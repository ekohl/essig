\setbeamertemplate{block begin}{
  \vskip.75ex
  \begin{beamercolorbox}[rounded=true,shadow=true,leftskip=1cm,colsep*=.75ex]{block title}%
    \usebeamerfont*{block title}\insertblocktitle
  \end{beamercolorbox}%
  {\ifbeamercolorempty[bg]{block body}{}{\nointerlineskip\vskip-0.5pt}}%
  \usebeamerfont{block body}%
  \begin{beamercolorbox}[rounded=true,shadow=true,colsep*=.75ex,sep=.75ex, vmode]{block body}%
    \ifbeamercolorempty[bg]{block body}{\vskip-.25ex}{\vskip-.75ex}\vbox{}%
    \begin{adjustwidth}{0.8cm}{0.4cm}
  }
  \setbeamertemplate{block end}{
  \end{adjustwidth}
  \end{beamercolorbox}
}

\begin{block}{\large \smash{4. The process algebra prCRL}\vphantom{Introduction}}
We introduce the specification language \alert{prCRL}, give by
\begin{align*}\color{black}
     p ::= Y(\dv) \ \mid\  c \Rightarrow p \ \mid \ p + p \ \mid \ \sum_{x:D} p \ \mid \ a(\dv) \!\!\psum_{x :D} f \colon p
\end{align*}
where $c$ is a \alert{condition}, $a$ an atomic \alert{action}, $f$ a \alert{real-valued expression} yielding values in $[0,1]$, and $\dv$ a \alert{vector of expressions}. 

\vskip20pt

\begin{adjustwidth}{0.5cm}{0.5cm}
\begin{itemize}
\item \ Based on $\mu$CRL (so \alert{data}), with additional \alert{probabilistic choice}
\item \ Operational semantics defined in terms of \alert{probabilistic automata}
\item \ Minimal set of operators to \alert{facilitate formal manipulation}
\item \ \alert{Syntactic sugar} easily definable
\end{itemize}
\end{adjustwidth}

\vskip20pt

\textit{Example:} $X = \tau \psum_{n:\mathbb{N}} \frac{1}{2^n}\colon \text{\it send}(n) \cdot X$. This specification repeatedly chooses a natural number $n$ with probability $\frac{1}{2^n}$, and then sends the number.
\end{block}	
