\setbeamertemplate{block begin}{
  \vskip.75ex
  \begin{beamercolorbox}[rounded=true,shadow=true,leftskip=1cm,colsep*=.75ex]{block title}%
    \usebeamerfont*{block title}\insertblocktitle
  \end{beamercolorbox}%
  {\ifbeamercolorempty[bg]{block body}{}{\nointerlineskip\vskip-0.5pt}}%
  \usebeamerfont{block body}%
  \begin{beamercolorbox}[ht=17cm, rounded=true,shadow=true,colsep*=.75ex,sep=.75ex, vmode]{block body}%
    \ifbeamercolorempty[bg]{block body}{\vskip-.25ex}{\vskip-.75ex}\vbox{}%
    \begin{adjustwidth}{0.8cm}{0.4cm}
  }
  \setbeamertemplate{block end}{
  \end{adjustwidth}
  \end{beamercolorbox}
}

\begin{block}{\large \smash{3. Our approach; an overview}\vphantom{Introduction}}
\textbf{Main idea:} we introduce a process algebra \alert{prCRL}, incorporating both \alert{data types} and \alert{probabilistic choice}. It has a \alert{linear format} (the \alert{LPPE}), enabling symbolic optimisations at the \alert{language level}. Therefore, the state space can be reduced \alert{before} it is generated.
\vskip47pt
\begin{figure}[h!]
\begin{center}
\begin{tikzpicture}[scale=1, transform shape]
	\node[smallrectangle] (s_0) {Probabilistic Specification (prCRL)};
	\node[] (temp) [right of=s_0, node distance=8cm] {};
	\node[smallrectangle] (s_2) [right of=temp, node distance=8cm] {State Space (PA)};
	\node[smallrectangle] (s_1) [below of=temp, node distance=5.5cm] {Linear Probabilistic Process  
Equation (LPPE)};
	\draw[->] (s_0) -- node [auto, swap, pos=0.2] {\color{red} Linearisation} (s_1);
	\draw[->] (s_1) -- node [auto, swap, pos=0.8] {\color{red} Instantiation} (s_2);
	
       %\node[] (s_3) [right of=s_0, node distance=11cm] {\parbox{7cm}{\begin{itemize}\item Probability \item Data \end{itemize}}};
%	\node[] (s_3) [right of=s_0, node distance=4cm] {\parbox{5cm}{\color{blue} $X(n:\mathbb{N}) = write(n) \cdot X(n+1)$\\$Y = 0.9 \colon \text{beep} \cdot Y \oplus 0.1 \colon \text{crash}$}};

\draw[->] (s_1) edge    [auto, loopleft] node %[star, draw=black]
 {\color{greenish} \bf Optimisation} ();
%	\node[] (s_8') [below left of=s_2, node distance=5.5cm] {};
%	\node[state, oval, fill=white] (s_8) [left of=s_8', node distance=2.25cm]  {\color{black} $\mathrlap{\text{\phantom{..}Visualisation}}$\phantom{Model checking}} ;
%	\node[] (s_9') [below right of=s_2, node distance=5.5cm] {};
%	\node[state, oval, fill=white] (s_9) [right of=s_9', node distance=2.25cm]  {\color{black} Model checking} ;
%	\draw[->] (s_2) -- node [auto] {} (s_8);
%	\draw[->] (s_2) -- node [auto] {} (s_9);
	
\end{tikzpicture}
\end{center}
\end{figure}
\end{block}	
